% rubber: set program xelatex
\documentclass[a4paper,12pt]{article}

\usepackage{xltxtra}
\usepackage{amsmath,amsthm,amssymb}

\usepackage[autostyle=true]{csquotes}
\usepackage{polyglossia}
\setmainlanguage{english}
\setotherlanguage[variant=mono]{greek}
\usepackage[xetex]{hyperref}

% Fonts
\setmainfont{CMU Serif}                 
\setsansfont{CMU Sans Serif}            
\setmonofont{CMU Typewriter Text}


\usepackage[margin=1in]{geometry} 

% Source code listings
\usepackage{listings}
\usepackage{color}
\usepackage{matlab-prettifier}
\lstdefinestyle{My-Matlab}
{
  style               = MatlabBaseStyle@mlpr,
  basicstyle          = \color{black}\ttfamily\small,
  mllastelementstyle  = \color{black}                    ,
  mlkeywordstyle      = \color[RGB]{000,000,255}         ,
  mlcommentstyle      = \color[RGB]{034,139,034}         ,
  mlstringstyle       = \color[RGB]{160,032,240}         ,
  mlsyscomstyle       = \color[RGB]{178,140,000}         ,
  mlsectiontitlestyle = \commentStyle@mlpr      \bfseries,
  mlsharedvarstyle    = \color[RGB]{000,163,163}         ,
  mlplaceholderstyle  = \mleditorphstyle,
}
% Disable paragraph intention 
\setlength\parindent{0pt}  

% Verbatim
\usepackage{fancyvrb}

\begin{document}
\title{ Multi-Sensors Fusion and Tracking\\Vibot Program 2018-2019}
\author{Ioannis Tsagatakis\\
1\textsuperscript{st} Assignment: Background Subtraction} 
 
\maketitle

I haven't manage to implement all methods in the given amount of time and with my limited computational resources. The code and the result videos van be found in the following gtihub repository \url{https://github.com/jtsagata/image_bg_removal}.

\section{Data Preperation}
To speed up development the image files are read only once and saved as \texttt{'mat'} files. The code for the \texttt{'gtruth.mat'} file is given bellow. The code for the other image sequences is simmilar and it is on the github repository.

\lstinputlisting[style=My-Matlab, firstline=15]{../Data/prepare_data_gtruth.m}

%% 
%% FRAME DIFFERENCING
%%
\section{Frame differencing}
The following matlab function implements the \textit{Frame differencing} method.

\lstinputlisting[style=My-Matlab]{../Code/bgsub_frame_diff.m}

To demonstrate we need a fuction to annotate the video with data like frame number and location, and with a bounding box around the main moving object of the scene.

\begin{figure}[Ht]
\centering
\includegraphics{../Videos/bgsub_framediff_cars.png}
\label{fig:fdiff_cars}
\caption{Frame differencing for the cars sequence}
\end{figure}

\lstinputlisting[style=My-Matlab]{../Code/anotate_video_box.m}

And a code to save the video sequence as a playable video on hard disk.

\lstinputlisting[style=My-Matlab]{../Code/videoSave.m}

Finnaly a fuction is created to save some frames in png file format.
\lstinputlisting[style=My-Matlab]{../Code/video_to_img_seq.m}

\subsection{Cars Sequence}
Let's apply the algorithm to the \texttt{'cars'} sequence.
\lstinputlisting[style=My-Matlab]{../Code/bgsub_demo_cars_fdiff.m}

Some frames is given in Figure~\ref{fig:fdiff_cars} on page \pageref{fig:fdiff_cars}.

%%
%%
%%
\subsection{Highway Sequence}

The highway sequence is in color. So we modify the algorithm. An Eucledian metric is used  to estimate the threshold of a given pixel form the background. Also the confusion matrix is calculated. After many revisions the final code is given bellow

\vspace{10pt}
\lstinputlisting[style=My-Matlab]{../Code/highway_frame_diff.m}
\vspace{30pt}

Some frames is given in Figure~\ref{fig:fdiff_h} on page \pageref{fig:fdiff_h}.

\begin{figure}[Ht]
\centering
\includegraphics{../Videos/highway_frame_diff.png}
\label{fig:fdiff_h}
\caption{Frame differencing for the highway sequence}
\end{figure} 


\vspace{10pt}

The precission, recall an F-score of the method is given bellow:

\VerbatimInput{../Videos/highway_frame_diff.txt}

%% 
%% RUNNING AVERAGE
%%
\section{The Running Average Gaussian Method}

\subsection{Cars Sequence}

The code that implements the Running Average Gaussian Method on the cars sequence is given bellow 

\vspace{10pt}
\lstinputlisting[style=My-Matlab]{../Code/a3_gaussian.m}
\vspace{30pt}

Some frames is given in Figure~\ref{fig:a3_gaussian_cars} on page \pageref{fig:a3_gaussian_cars}.

\begin{figure}[Ht]
\centering
\includegraphics{../Videos/a3_gaussian.png}
\label{fig:a3_gaussian_cars}
\caption{Running Average Gaussian for the cars sequence}
\end{figure}
 

\subsection{Highway Sequence}

Bellow is the code that implements the running average gaussian method for the highway sequence:
\vspace{10pt}
\lstinputlisting[style=My-Matlab]{../Code/highway_avg_gauss.m}
\vspace{30pt}

Some frames is given in Figure~\ref{fig:faverageh} on page \pageref{fig:faverageh}.

\begin{figure}[Ht]
\centering
\includegraphics{../Videos/highway_avg_gauss.png}
\label{fig:faverageh}
\caption{Running Average Gaussian for the highway sequence}
\end{figure} 


\vspace{10pt}

The precission, recall an F-score of the method is given bellow:

\VerbatimInput{../Videos/highway_avg_gauss.txt}

%% 
%% EIGEN BACKGROUND
%% 
\section{Eigen Background}
Code for the eigen background method (slow, memory hog, not working).

\vspace{10pt}
\lstinputlisting[style=My-Matlab]{../Code/a5_eigen.m}
\vspace{30pt}



\vfill
\noindent Tools used: \texttt{Matlab}, \LaTeX{}.
\end{document}
